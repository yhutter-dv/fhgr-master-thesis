\chapter*{Abstract}
Diese Masterarbeit befasst sich mit der Konzeption und Implementierung einer webbasierten, echtzeitfähigen 3D-Terrainvisualisierung auf Basis von Geodaten des \acrfull{swisstopo}. Ziel der Arbeit ist es, grossflächige Gebirgslandschaften interaktiv im Browser darzustellen und dabei sowohl technische als auch ästhetische Aspekte der Visualisierung zu untersuchen.

Als Datengrundlage werden die Datensätze swissALTI3D für das digitale Höhenmodell sowie swissIMAGE mit orthografisch korrigierten Luftbildern verwendet. Zur Verarbeitung dieser Daten wurde eine weitgehend automatisierte Datenvorverarbeitung entwickelt, die den Download, das Zusammenfügen, die Normalisierung sowie die Aufteilung der Daten in mehrere Detailstufen (Level of Detail) umfasst. Die eigentliche Visualisierung basiert auf dem Three.js-Framework und nutzt einen Quadtree-Algorithmus zur dynamischen Unterteilung des Terrains in Abhängigkeit von der Kameraposition.

Ein zentraler Fokus der Arbeit liegt auf der Identifikation und Lösung typischer Herausforderungen bei der 3D-Terrainvisualisierung, darunter sichtbare Risse zwischen unterschiedlichen Detailstufen sowie performanzkritische Engpässe bei Rendering und Datenverarbeitung. Diese Probleme werden durch Techniken wie Index Stitching, Vorladen von Texturen, Multithreading mittels Web Workern sowie optimierte Texturformate adressiert. Ergänzend werden verschiedene ästhetische Gestaltungsmöglichkeiten untersucht, etwa High Dynamic Range, Tone Mapping sowie Anisotropic Filtering. Die Ergebnisse zeigen, dass eine performante Terrainvisualisierung auch im Webbrowser realisierbar ist. Die entwickelte Lösung erreicht Bildraten von mindestens 60 FPS und eignet sich als Grundlage für weiterführende Anwendungen, beispielsweise im Kontext von CAVE-Systemen, oder Virtual Reality.