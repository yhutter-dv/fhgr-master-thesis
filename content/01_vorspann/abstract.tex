\chapter*{Abstract}
Diese Masterarbeit befasst sich mit der Konzeption und Implementierung einer webbasierten, echtzeitfähigen 3D-Terrainvisualisierung auf Basis von Geodaten des Bundesamts für Landestopografie (swisstopo). Ziel der Arbeit ist es, grossflächige Gebirgslandschaften interaktiv im Browser darzustellen und dabei sowohl technische als auch ästhetische Aspekte der Visualisierung zu untersuchen.

Als Datengrundlage dienen das digitale Höhenmodell swissALTI3D sowie die orthografisch korrigierten Luftbilder des swissIMAGE-Datensatzes. Zur Verarbeitung dieser Daten wurde eine automatisierte Datenvorverarbeitung entwickelt, die den Download, das Zusammenführen, die Normalisierung sowie die Aufteilung der Daten in mehrere Detailstufen (Level of Detail) umfasst. Die eigentliche Visualisierung basiert auf dem Three.js-Framework und nutzt einen Quadtree-Algorithmus zur dynamischen Unterteilung des Terrains in Abhängigkeit von der Kameraposition.

Ein zentraler Fokus der Arbeit liegt auf der Identifikation und Lösung typischer Herausforderungen der 3D-Terrainvisualisierung, darunter sichtbare Risse zwischen unterschiedlichen Detailstufen sowie leistungslrelevante Engpässe bei Rendering und Datenverarbeitung. Diese Probleme werden durch Techniken wie Index Stitching, Vorladen von Texturen, Multithreading mittels Web Workern sowie das Verwenden von optimierten Texturformaten adressiert. Ergänzend werden verschiedene ästhetische Gestaltungsmöglichkeiten wie High Dynamic Range und Tone Mapping untersucht. Die Ergebnisse zeigen, dass eine echtzeitfähige Terrainvisualisierung auch im Webbrowser realisierbar ist. Die entwickelte Lösung eignet sich als Grundlage für weiterführende Anwendungen, beispielsweise im Kontext von CAVE-Systemen, oder Virtual Reality.