\chapter{Datengrundlage}
\label{chap_datengrundlage}
Als Datengrundlage dienen die Gebirgsdaten vom \acrfull{swisstopo}. swisstopo bietet hierzu zwei relevante Datensätze an. Der \textbf{swissALTI3D-Datensatz} beinhaltet ein detailliertes Höhenmodell (ohne Bewuchs und Bebauung) und wird alle \textbf{6 Jahre} aktualisiert \parencite{swissALTI3D_2024}. Der \textbf{swissIMAGE-Datensatz} hingegen enthält orthografisch korrigierte Luftbilder der Schweizer Gebirge und wird in einem Zyklus von \textbf{3 Jahren} aktualisiert \parencite{swissIMAGE_2024}. Die Kombination dieser zwei Datensätze erlaubt es, die Gebirgsketten in 3D zu rekonstruieren. Beide Datensätze können kostenlos von der swisstopo Webseite\footnote{\url{https://www.swisstopo.admin.ch}} heruntergeladen werden. 

\section{swissALTI3D Datensatz}
Der swissALTI3D Datensatz beinhaltet die \textbf{Höhendaten der Gebirgsketten} sowohl in der Schweiz als auch des Fürstentums Liechtenstein (siehe Abbildung \ref{fig_swissALTI3D}). Je nach Gebirgshöhe kommen unterschiedliche Messtechniken zum Einsatz. Gebirgsketten unterhalb von 2000m ü.M. werden mit Lasermesstechniken wie \acrfull{LiDAR} vermessen, welche eine Genauigkeit von bis zu \pm30cm ermöglichen. Bei Gebieten oberhalb von 2000m ü.M  wird Stereokorrelation mit einer Messgenauigkeit von \pm1m - 3m eingesetzt \parencite[S. 8]{swissALTI3D_info_2024}. 

\begin{figure}[H]
    \caption{Höhendatenabdeckung von swissALTI3D \parencite[S. 4]{swissALTI3D_info_2024}}
    \includegraphics[width=.5\linewidth]{content/00_assets/swissALTI3D.png}
    \label{fig_swissALTI3D}
\end{figure}

Die Daten werden in Form von km$^2$-Kacheln in jeweils zwei unterschiedlichen Datenformaten zur Verfügung gestellt. Das \textbf{\acrfull{COG} Format} beinhaltet nebst den Höhenwerten auch Informationen zur Georeferenzierung. Die einzelnen Pixel der Bilddatei repräsentieren die Höhenwerte und sind in einer 32-bit Auflösung verfügbar.  Anders als das \acrshort{COG} Format werden die Höhenwerte im ASCII-Format nicht als Bild, sondern als Textdatei mit Zahlenwerten gespeichert. Für jeden Wert wird hierbei eine Genauigkeit von 4 Nachkommastellen garantiert \parencite[S. 4-5]{swissALTI3D_info_2024}. 

Der Datensatz ist in zwei verschiedenen Bodenauflösungen, einmal mit 50 Zentimetern pro Pixel sowie 2 Meter pro Pixel, vorhanden. Tabelle \ref{table:datenverbrauch_swissALTI3D} zeigt den notwendigen Speicherplatz für den gesamten Datensatz, aufgeschlüsselt nach Auflösung und Datenformat. Der Datensatz wird alle 6 Jahre komplett aktualisiert, entsprechend wird pro Jahr $\frac{1}{6}$ des Gesamtdatensatzes neu erfasst \parencite[S. 6]{swissALTI3D_info_2024}.

\begin{table}[H]
    \caption{Speicherbedarf swissALTI3D Datensatzes aufgeschlüsselt nach Auflösung und Datenformat \parencite[S. 6]{swissALTI3D_info_2024}}
    \begin{tabularx}{\textwidth} {
        >{\raggedright\arraybackslash}X 
        >{\raggedright\arraybackslash}X
        >{\raggedright\arraybackslash}X}
            \hline
            \textbf{Auflösung} & \textbf{GeoTIFF} & \textbf{ASCII XYZ}  \\
            \hline
            50cm     & 800 GB & 5 TB     \\
            2m       & 50.5 GB & 308 GB     \\
            \hline
    \end{tabularx}
    \bigbreak
    \label{table:datenverbrauch_swissALTI3D}
\end{table}

\section{swissIMAGE Datensatz}
Der swissIMAGE Datensatz beinhaltet orthografische Luftbilder. Die Daten sind in drei unterschiedlichen Auflösungen  vorhanden. Je nach Erfassungszeitraum ist eine Auflösung von bis zu 10 Zentimeter pro Pixel möglich (siehe Tabelle \ref{table:aufloesung_swissIMAGE}) 

\begin{table}[H]
    \caption{Maximale Bodenauflösung des swissIMAGE Datensatzes aufgeschlüsselt nach Jahr der Aufnahme \parencite[S. 3]{swissIMAGE_info_2024}}
    \begin{tabularx}{\textwidth} {
        >{\raggedright\arraybackslash}X 
        >{\raggedright\arraybackslash}X}
            \hline
            \textbf{Erfassungszeitraum} & \textbf{Maximale Bodenauflösung pro Pixel}  \\
            \hline
            1998 bis 2005 & 50cm \\
            2005 bis 2016 & 25cm \\
            Ab 2017 & 10cm \\
            \hline
    \end{tabularx}
    \bigbreak
    \label{table:aufloesung_swissIMAGE}
\end{table}

Um eine einheitliche Datengrundlage zu gewährleisten, werden die swissIMAGE Daten von swisstopo ebenfalls als einzelne km$^2$-Kacheln zur Verfügung gestellt. Beide Datensätze orientieren sich hierbei am LV95-Koordinatensystem der Schweiz (siehe Abbildung \ref{fig_dataset_lv95}). Der Speicherbedarf pro Kachel hängt von der entsprechenden Auflösung ab. Kacheln mit einer Auflösung von 2 Meter pro Bildpunkt benötigen rund 0.2MB Speicherbedarf. Für die höchste Auflösung von 10 Zentimetern sind es hingegen bereits 60MB. Der swissIMAGE Datensatz wird in einem Dreijahreszyklus aktualisiert \parencite[S. 8 + S. 9]{swissIMAGE_info_2024}.
\begin{figure}[H]
    \caption{Ausrichtung der Datensätze am LV95-Koordinatensystem \parencite[S. 4]{swissIMAGE_info_2024}}
    \includegraphics[width=.5\linewidth]{content/00_assets/swissIMAGE_lv95.png}
    \label{fig_dataset_lv95}
\end{figure}

\section{LV95 (EPSG 2056) Koordinatensystem}
Beide oben thematisierten Datensätze orientieren sich am Schweizer Landeskoordinatensystem, auch bekannt unter den Begriffen LV95 sowie EPSG 2056. LV95 bezieht sich hierbei auf den Zeitpunkt der Landesvermessung im Jahr 1995. Das Koordinatensystem selbst besitzt zwei Achsen, die Nord (N) sowie die Ostachse (E) (siehe Abbildung \ref{fig_lv95}). Der Ursprung des Koordinatensystems befindet sich bei der alten Sternwarte in Bern (N = 1'200'000m / E = 2'600'000m). Der Ursprung wurde bewusst nicht bei N = 0m / E = 0m gewählt, um negative Koordinaten sowohl südlich als auch westlich von Bern zu vermeiden. Ebenso ist eine Verwechslung zwischen Ost- und Nordwerten damit ausgeschlossen. Zur Unterscheidung vom vorherigen Koordinatensystem (LV03) werden der Nordkoordinate 1'000'000m und der Ostkoordinate  2'000'000m hinzugefügt. Dies hat den positiven Nebeneffekt, dass mit zwei siebenstelligen Zahlen jeder Punkt in der Schweiz genau angegeben werden kann \parencite{swisstopo_lv95_2025}.
\begin{figure}[H]
    \caption{LV95-Koordinatensystem \parencite{swisstopo_lv95_2025}}
    \includegraphics[width=.7\linewidth]{content/00_assets/lv95_koordinatensystem.png}
    \label{fig_lv95}
\end{figure}
