\chapter{Datengrundlage}
\label{chap_datengrundlage}
Als Datengrundlage wurden die Gebirgsdaten vom \acrfull{swisstopo} ausgewählt. Hierfür bietet \acrshort{swisstopo} zwei verschiedene Datensätze an. Der swissALTI3D-Datensatz beinhaltet ein detailliertes 3D Höhenmodell (ohne Bewuchs und Bebauung) zur Verfügung. Dieser Datensatz bietet eine ideale Grundlage für 3D Visualisierungen oder auch Simulationen. Der swissALTI3D Datensatz wird alle 6 Jahre aktualisiert \parencite{swissALTI3D_2024}. Der swissIMAGE-Datensatz hingegen enthält ortographisch korrigierte Luftbilder der schweizer Gebirge und wird in einem Zyklus von 3 Jahren aktualisiert \parencite{swissIMAGE_2024}. Kombiniert man diese zwei Datensätze, so lassen sich die Gebirge in 3D rekonstruieren. Beide Datensätze können kostenlos von der swisstopo Webseite\footnote{\url{https://www.swisstopo.admin.ch/}} heruntergeladen werden. 

\section{swissALTI3D-Datensatz}
Der swissALTI3D Datensatz beinhaltet die Höhendaten der Gebirgsketten sowohl in der Schweiz als auch des Fürstentums Liechtenstein (siehe Abbildung \ref{fig_swissALTI3D}). 

\begin{figure}[H]
    \caption{Höhendatenabdeckung von swissALTI3D \parencite[S. 4]{swissALTI3D_info_2024}}
    \includegraphics[width=.5\linewidth]{content/00_assets/swissALTI3D.png}
    \label{fig_swissALTI3D}
\end{figure}


Je nach Gebirgshöhe kommen unterschiedliche Messtechniken zum Einsatz. Gebirgsketten unterhalb von 2000m ü.M. werden mit Lasermesstechniken wie \acrfull{LiDAR} vermessen, was eine Genauigkeit von  \pm30cm ermöglicht. Bei Gebiete oberhalb von 2000m ü.M  wird Stereokorrelation verwendet, was eine Genauigkeit von \pm1m - 3m ermöglicht \parencite[S. 8]{swissALTI3D_info_2024}. 


Die Daten selbst werden in Form von km$^2$-Kacheln zur Verfügung gestellt. Es stehen hierbei zwei verschiedene Datenformate zur Verfügung. Das \acrfull{COG} Format beinhaltet nebst den Rasterdaten auch Informationen zur Georeferenzierung. Jeder Pixel besitzt hierbei einen Höhenwert und wird als 32bit floating point (Gleitkommazahl) dargestellt. Dieses Datenformat ist besonders für cloud-services optimiert. Das ASCII X, Y, Z single space Format referenziert den Höhenwert immer im Mittelpunkt der Rasterzelle mit einer garantierten Genauigkeit von 4 Nachkommastellen \parencite[S. 4-5]{swissALTI3D_info_2024}. Der swissALTI3D Datensatz ist in zwei verschiedenen Bodenauflösungen (50cm sowie 2m pro Pixel) vorhanden. Tabelle \ref{table:datenverbrauch_swissALTI3D} zeigt den notwendigen Speicherplatz für den gesamten Datensatz aufgeschlüsselt nach Auflösung und Datenformat. Der Datensatz wird in einem 6 Jahreszyklus komplett aktualisiert, sprich pro Jahr wird rund $\frac{1}{6}$ des Gesamtdatensatzes auf Stand gebracht \parencite[S. 6]{swissALTI3D_info_2024}.

\begin{table}[H]
    \caption{Speicherbedarf swissALTI3D Datensatz aufgeschlüsselt nach Auflösung und Datenformat \parencite[S. 6]{swissALTI3D_info_2024}}
    \begin{tabularx}{\textwidth} {
        >{\raggedright\arraybackslash}X 
        >{\raggedright\arraybackslash}X
        >{\raggedright\arraybackslash}X}
            \hline
            \textbf{Auflösung} & \textbf{GeoTIFF} & \textbf{ASCII XYZ}  \\
            \hline
            50cm     & 800 GB & 5 TB     \\
            2m       & 50.5 GB & 308 GB     \\
            \hline
    \end{tabularx}
    \bigbreak
    \label{table:datenverbrauch_swissALTI3D}
\end{table}

\section{swissIMAGE-Datensatz}
Der swissIMAGE-Datensatz beinhaltet ortographisch (verzerrungsfreie) korrigierte Luftbilder. Die Daten selbst sind in drei unterschiedlichen Auflösungen (siehe Tabelle \ref{table:aufloesung_swissIMAGE}) vorhanden. Die Bodenauflösung des neusten Datensatzes (ab 2017) liegt hierbei zwischen 10cm bis 2m.
\begin{table}[H]
    \caption{Maximale Bodenauflösung des swissIMAGE Datensatzes aufgeschlüsselt nach Jahr der Aufnahme \parencite[S. 3]{swissIMAGE_info_2024}}
    \begin{tabularx}{\textwidth} {
        >{\raggedright\arraybackslash}X 
        >{\raggedright\arraybackslash}X}
            \hline
            \textbf{Erfassungszeitraum} & \textbf{Maximale Bodenauflösung pro Pixel}  \\
            \hline
            1998 bis 2005 & 50cm \\
            2005 bis 2016 & 25cm \\
            Ab 2017 & 10cm \\
            \hline
    \end{tabularx}
    \bigbreak
    \label{table:aufloesung_swissIMAGE}
\end{table}

Auch bei swissIMAGE werden die Daten als einzelne km$^2$-Kacheln zur Verfügung gestellt. Dies erleichtert die Kombination mit dem swissALTI3D Datensatz. Die Kacheln selbst orientieren sich ebenfalls wie bei swissALTI3D am LV95-Koordinatensystem der Schweiz (siehe Abbildung \ref{fig_swissIMAGE_lv95}). Der Speicherbedarf pro Kachel beträgt hierbei 0.2MB/km$^2$ für eine Auflösung von 2m sowie 60MB/km$^2$ für die maximale Auflösung von 10cm. Der swissIMAGE Datensatz wird in einem 3 Jahreszyklus aktualisiert. Pro Jahr wird somit rund $\frac{1}{3}$ des Gesamtdatensatzes überarbeitet \parencite[S. 8 + S. 9]{swissIMAGE_info_2024}.
\begin{figure}[H]
    \caption{Ausrichtung der swissIMAGE Kacheln am LV95-Koordinatensystem \parencite[S. 4]{swissIMAGE_info_2024}}
    \includegraphics[width=.5\linewidth]{content/00_assets/swissIMAGE_lv95.png}
    \label{fig_swissIMAGE_lv95}
\end{figure}

\section{LV95 (EPSG 2056) Koordinatensystem}
Beide oben thematisierte Datensätze orientieren sich am schweizer Landeskoordinatensystem auch unter dem Begriff LV95 oder EPSG 2056 bekannt. LV95 bezieht sich hierbei auf den Zeitpunkt der Landesvermessung (1995). Das Koordinatensystem selbst besitzt zwei Achsen, die Nord (N) sowie die Ostachse (E) (siehe Abbildung \ref{fig_lv95}). Der Ursprung des Koordinatensystems liegt hierbei bei der alten Sternwarte in Bern und hat die Koordinaten N = 1'200'000m / E = 2'600'000m. Das der Ursprung nicht die Koordinate 0m / 0m hat, mag auf den ersten Blick komisch erscheinen, das liegt jedoch daran, dass somit keine negativen Koordinaten sowohl südlich als auch westlich von Bern möglich sind. Ebenso ist eine Verwechslung zwischen Ost- und Nordwerten damit ausgeschlossen. Damit eine Unterscheidung zum vorherigen Koordinatensystem (LV03) möglich ist, erhielt die Nordkoordinate einen Zuschlag von 1'000'000m und die Ostkoordinate einen Zuschlag von 2'000'000m. Dies hat zudem auch noch den Nebeneffekt, dass mit zwei siebenstelligen Zahlen jeder Punkt in der Schweiz genau angegeben werden kann \parencite{swisstopo_lv95_2025}.
\begin{figure}[H]
    \caption{LV95-Koordinatensystem \parencite{swisstopo_lv95_2025}}
    \includegraphics[width=.7\linewidth]{content/00_assets/lv95_koordinatensystem.png}
    \label{fig_lv95}
\end{figure}
