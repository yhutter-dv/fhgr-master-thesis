\chapter{Einleitung}
\label{chap_einleitung}
In der heutigen Zeit gibt es viele verschiedene Möglichkeiten, Datenvisualisierungen auf interaktive Art und Weise zugänglich zu machen. Nebst der Darstellungsdimensionalität (2D/3D) spielt auch die eingesetzte Technologie eine entscheidende Rolle. Mit dem Aufkommen von immersiven Technologien wie \acrfull{VR} und \acrfull{CAVE} gibt es viele Optionen, die Daten für den Nutzer explorierbar zu machen. Der Fokus der vorliegenden Masterarbeit ist die Darstellung von Daten im dreidimensionalen Raum, wobei ein immersives Erlebnis im Vordergrund steht. Konkret hat sich die Arbeit zum Ziel gesetzt, eine 3D Terrain Visualisierung auf Basis von swisstopo-Daten für das \acrshort{CAVE}-System der \acrfull{FHGR} zu implementieren. Die Visualisierung soll hierbei in Echtzeit exploriert werden können. Im Rahmen der Arbeit sollen folgende Forschungsfragen geklärt werden:
\begin{itemize}
    \item Welche Technologie ist für eine echtzeitfähige 3D Datenvisualisierung geeignet?
    \item Welche Algorithmen sind für eine echtzeitfähige 3D Datenvisualisierung von Gebirgen geeignet?
    \item Welche Probleme treten bei der Datenvorverarbeitung auf?
    \item Welche Probleme treten bei der 3D Datenvisualisierung von Gebirgen auf?
    \item Wie kann die Ästhetik einer 3D Datenvisualisierung beeinflusst werden?
    \item Welche Optimierungen sind notwendig, um die Echtzeitfähigkeit der Visualisierung zu gewährleisten?
\end{itemize}

Der thematische Aufbau der Arbeit orientiert sich an den oben stehenden Forschungsfragen und gliedert sich wie folgt auf. Kapitel \ref{chap_datengrundlage} befasst sich mit den zugrundeliegenden Daten der Visualisierung. Kapitel \ref{chap_technologien} zeigt auf welche bestehenden Technologien es gibt, um Daten auf effiziente Weise in 3D darzustellen. Kapitel \ref{chap_render_pipelines} erläutert wichtige Terminologien und schafft ein notwendiges Grundverständnis, wie 3D-Rendering im Allgemeinen funktioniert. Kapitel \ref{chap_algorithmen} veranschaulicht wichtige Algorithmen, um die Echtzeitfähigkeit zu gewährleisten. In Kapitel \ref{chap_swiss_terrain_3d}  geht es um die eigentliche Implementierung der Terrainvisualisierung. Hierbei wird auf wichtige Aspekte wie die Datenvorverarbeitung, die genutzten Algorithmen, auftretende Problematiken sowie Optimierungen eingegangen. Anschliessend erfolgt in Kapitel \ref{chap_diskussion} eine Diskussion der Arbeit mit Bezug auf die oben definierten Forschungsfragen. Abgerundet wird die Arbeit durch Kapitel \ref{chap_ausblick}, welches einen Ausblick über mögliche Erweiterungen der Visualisierung und nicht thematisierte Aspekte behandelt.
