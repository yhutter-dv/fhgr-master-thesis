\chapter{Einleitung}
\label{chap_einleitung}
In der heutigen Zeit gibt es verschiedene Möglichkeiten, Datenvisualisierungen und die notwendigen Interaktionen interessant zu gestalten. Nebst der Dimension (2D und 3D) spielt auch die eingesetzte Technologie eine entscheidende Rolle. Mit dem Aufkommen von immersiven Technologien wie \acrfull{VR} und \acrfull{CAVE} gibt es viele Optionen, die Daten für Personen auf interaktive Art analysierbar zu machen.

Die vorliegende Masterarbeit fokussiert sich hierbei auf die Darstellung von Daten im dreidimensionalen Raum, wobei ein immersives Erlebnis im Vordergrund steht. Konkret hat sich die Arbeit zum Ziel gesetzt, eine 3D-Terrainvisualisierung auf Basis von swisstopo-Daten zu implementieren. Das Terrain soll hierbei in Echtzeit exploriert werden können. Damit aus den Daten eine Visualisierung erstellt werden kann, sind verschiedene Prozesse notwendig. In der Datenvisualisierung werden diese Prozesse in Form einer Data Visualization Pipeline zusammengefasst (siehe Abbildung \ref{fig_data_visualization_pipeline}).

\begin{figure}[H]
    \caption{Beispiel Data Visualization Pipline \parencite{data_visualization_pipeline_2006}}
    \includegraphics[width=.7\linewidth]{content/00_assets/data_visualization_pipeline.png}
    \label{fig_data_visualization_pipeline}
\end{figure}

Das Vorgehensmodell der Arbeit orientiert sich ebenfalls an einer Data Visualization Pipeline, wodurch sich in den einzelnen Prozessschritten folgende Forschungsfragen ergeben:
\begin{itemize}
    \item Welche Technologie ist für eine echtzeitfähige 3D Datenvisualisierung geeignet?
    \item Welche Algorithmen sind für eine echtzeitfähige 3D Datenvisualisierung von Gebirgen geeignet?
    \item Welche Probleme treten bei der Datenvorverarbeitung auf?
    \item Welche Probleme treten bei der 3D Datenvisualisierung von Gebirgen auf?
    \item Wie kann die Ästhetik einer 3D Datenvisualisierung beeinflusst werden?
    \item Welche Optimierungen sind notwendig, um die Echtzeitfähigkeit der Visualisierung zu gewährleisten?
\end{itemize}

Der thematische Aufbau der Arbeit, gestützt auf die oben definierten Forschungsfragen, gliedert sich wie folgt auf. Kapitel \ref{chap_datengrundlage} befasst sich mit den zugrundeliegenden Daten der Visualisierung. Kapitel \ref{chap_technologien} zeigt auf, welche Technologien existieren, um Daten auf effiziente Weise in 3D darzustellen. Kapitel \ref{chap_render_pipelines} erläutert wichtige Terminologien und schafft ein notwendiges Grundverständnis, wie 3D-Rendering im Allgemeinen funktioniert. Kapitel \ref{chap_algorithmen} veranschaulicht wichtige Algorithmen, um die Echtzeitfähigkeit zu gewährleisten. In Kapitel \ref{chap_swiss_terrain_3d}  geht es um die eigentliche Implementierung der Terrainvisualisierung. Hierbei wird auf wichtige Aspekte wie die Datenvorverarbeitung, die genutzten Algorithmen, auftretende Problematiken sowie Optimierungen eingegangen. Anschliessend erfolgt in Kapitel \ref{chap_diskussion} eine Diskussion der Arbeit mit Bezug auf die oben definierten Forschungsfragen. Abgerundet wird die Arbeit durch Kapitel \ref{chap_ausblick}, welches mögliche Erweiterungen und nicht thematisierte Aspekte behandelt.
