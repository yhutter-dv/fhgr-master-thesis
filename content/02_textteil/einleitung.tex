\chapter{Einleitung}
\label{chap_einleitung}
In der heutigen Zeit existieren zahlreiche Möglichkeiten, Datenvisualisierungen sowie die damit verbundenen Interaktionen ansprechend zu gestalten. Neben der gewählten Darstellungsdimension (2D oder 3D) spielt insbesondere die eingesetzte Technologie eine zentrale Rolle. Mit dem Aufkommen immersiver Technologien wie \acrfull{VR} und \acrfull{CAVE} eröffnen sich vielfältige Ansätze, Daten auf interaktive und explorative Weise analysierbar zu machen.

Die vorliegende Masterarbeit konzentriert sich auf die Darstellung von Daten im dreidimensionalen Raum, wobei ein immersives Nutzungserlebnis im Vordergrund steht. Ziel der Arbeit ist die Implementierung einer echtzeitfähigen 3D-Terrainvisualisierung auf Basis von swisstopo-Daten, die eine interaktive Exploration des Terrains ermöglicht. Um aus den zugrunde liegenden Daten eine Visualisierung zu erzeugen, sind mehrere Verarbeitungsschritte erforderlich. In der Datenvisualisierung werden diese Schritte mittels einer \textbf{Data Visualization Pipeline} zusammengefasst (siehe Abbildung \ref{fig_data_visualization_pipeline}).

\begin{figure}[H]
    \caption{Beispiel einer Data Visualization Pipline \parencite{data_visualization_pipeline_2006}}
    \includegraphics[width=.7\linewidth]{content/00_assets/data_visualization_pipeline.png}
    \label{fig_data_visualization_pipeline}
\end{figure}

Das Vorgehensmodell der Arbeit orientiert sich ebenfalls an einer Data-Visualization-Pipeline, aus deren einzelnen Prozessschritten sich mehrere Forschungsfragen ableiten. Zunächst wird untersucht, welche Technologien sich für eine echtzeitfähige 3D-Datenvisualisierung eignen (F1) und welche Algorithmen für die Darstellung von Gebirgslandschaften in Echtzeit verwendet werden können (F2). Darüber hinaus werden sowohl die Herausforderungen der Datenvorverarbeitung (F3) als auch die spezifischen Problemstellungen der 3D-Datenvisualisierung von Gebirgen analysiert (F4). Ein weiterer Fokus liegt auf der Frage, wie die Ästhetik einer 3D-Datenvisualisierung gezielt beeinflusst werden kann (F5) und welche Optimierungen erforderlich sind, um die Echtzeitfähigkeit der Visualisierung sicherzustellen (F6).

Der thematische Aufbau der Arbeit orientiert sich an diesen Forschungsfragen und gliedert sich wie folgt: Kapitel \ref{chap_datengrundlage} befasst sich mit den zugrunde liegenden Datensätzen der Visualisierung. In Kapitel \ref{chap_technologien} werden verschiedene Technologien vorgestellt, die eine effiziente dreidimensionale Darstellung von Daten ermöglichen. Kapitel \ref{chap_render_pipelines} erläutert zentrale Terminologien und vermittelt ein grundlegendes Verständnis der Funktionsweise von 3D-Rendering. Aufbauend darauf werden in Kapitel \ref{chap_algorithmen} wichtige Algorithmen zur Gewährleistung der Echtzeitfähigkeit vorgestellt. Kapitel \ref{chap_swiss_terrain_3d} widmet sich der konkreten Implementierung der Terrainvisualisierung und behandelt dabei Aspekte wie Datenvorverarbeitung, eingesetzte Algorithmen, auftretende Herausforderungen sowie Optimierungsansätze. In Kapitel \ref{chap_diskussion} werden die Ergebnisse der Arbeit im Kontext der definierten Forschungsfragen diskutiert. Den Abschluss bildet Kapitel \ref{chap_ausblick}, das mögliche Erweiterungen sowie nicht behandelte Aspekte aufzeigt.
